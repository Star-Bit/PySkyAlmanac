\documentclass[12pt]{article}
\usepackage{graphicx}
\usepackage{amsmath}
\usepackage[utf8x]{inputenc}
\usepackage[T1]{fontenc}
\usepackage{url}
\usepackage{caption}
\usepackage{palatino}
\usepackage[paper=a4paper,margin=2cm,bmargin=3cm]{geometry}
%\usepackage[turkish]{babel}
\author{M. Atakan Gürkan, Alp Akoğlu}
\title{Çizelgenin Kullanımı}
\date{}
\begin{document}
\maketitle
Bu çizelge 2012 yılı için çeşitli gök cisimlerinin doğma,
meridyenden (gökyüzünde en yüksek noktasından) geçme ve batma
zamanlarını, alacakaranlığın sonuyla başlangıcını ve Ay’ın
evrelerini veriyor. Dikey eksen günleri, yatay eksen gece boyunca
zamanı gösteriyor. Gece içinde yarımşar saatlik aralıklar ve yıl
içinde pazar akşamlarını pazartesi sabahlarına bağlayan geceler
noktalı çizgilerle belirtiliyor. Düşey olarak iki nokta arası
bir güne, yatay olarak iki nokta arası beş dakikaya karşılık
geliyor.

Bu çizelge Ankara için hazırlanmıştır. Ancak ülkemizin
tamamında kullanılabilir. Yalnız gözlem yerinin Ankara’dan
uzaklığına bağlı olarak zamanlarda küçük farklar olacaktır.
Ankara’nın doğusundaki noktalarda çizelgede verilen olaylar daha
erken, batısındaki noktalarda daha geç olacaktır.

Not: Yaz saati uygulamasının geçerli olduğu zamanlarda çizelgede
okunan saatlere bir saat eklemek gerekiyor.

{\bfseries Örnek:}
Bir örnek olarak 16 Mart gecesinin olaylarına bakılırsa:
Öncelikle, sol tarafta 17 Mart’a karşılık gelen noktanın
üstünde yaklaşık olarak 16 Mart’a karşılık gelen noktayı
bulmak gerekiyor.  Buradan sağa doğru ilerlediğimizde 18:30
civarında Mars'ın batacağını görüyoruz; bu bilgi bize aynı
zamanda Güneş battığı zaman Mars'ın gökyüzünde olduğunu
da söylüyor. Mars'ın ardından 18:50 civarında Uranüs batacak,
19:55 civarında Polluks meridyenden geçecek, 21:25 civarında da
Satürn doğacak.  Sağa doğru ilerledikçe, belli
saatlerde pekçok gökcisminin doğduğunu, meridyenden geçtiğini
ve battığını gö\-rü\-yo\-ruz. 22:50'de gördüğümüz hilal simgesi
Ay'ın batış zamanını gösteriyor ve bir sonraki gün Ay'ın daha
büyük olacağını belirtiyor.  Son olarak 19:30 ve 4:25 civarında
gördüğümüz kesikli çizgiler sırasıyla alacakaranlığın
bitişini ve başlangıcını belirtiyor. Bu noktalar Güneş’in
ufkun 18° altında kaldığı anlara karşılık geliyor.

\end{document}
